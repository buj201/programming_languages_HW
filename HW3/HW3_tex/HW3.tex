%%%%%%%%%%%%%%%%%%%%%%%%%%%%%%%%%%%%%%%%%
% Short Sectioned Assignment
% LaTeX Template
% Version 1.0 (5/5/12)
%
% This template has been downloaded from:
% http://www.LaTeXTemplates.com
%
% Original author:
% Frits Wenneker (http://www.howtotex.com)
%
% License:
% CC BY-NC-SA 3.0 (http://creativecommons.org/licenses/by-nc-sa/3.0/)
%
%%%%%%%%%%%%%%%%%%%%%%%%%%%%%%%%%%%%%%%%%

%----------------------------------------------------------------------------------------
%	PACKAGES AND OTHER DOCUMENT CONFIGURATIONS
%----------------------------------------------------------------------------------------

\documentclass[paper=a4, fontsize=11pt]{scrartcl} % A4 paper and 11pt font size

\usepackage[T1]{fontenc} % Use 8-bit encoding that has 256 glyphs
%\usepackage{fourier} % Use the Adobe Utopia font for the document - comment this line to return to the LaTeX default
\usepackage[english]{babel} % English language/hyphenation
\usepackage{amsmath,amsfonts,amsthm} % Math packages
\usepackage{mathtools} %More math! (For dscases)
\usepackage{hyperref} %HTML package
\usepackage{pgfplots} %Makes plots in LaTeX
\usepackage{tikz} %Also tikz?
\usepackage{bm} %makes vectors bold
\usepackage{bbm} %Blackboard bold 1
\usepgfplotslibrary{fillbetween}%Let's me fill between named plots
\usepackage{graphicx} %import pics

\usepackage[noend]{algpseudocode} %Algorithms
\usepackage{algorithm}

\graphicspath{ {Python\_figs/} }
\DeclareGraphicsExtensions{.pdf,.png,.jpg}
\usepackage{sectsty} % Allows customizing section commands
\allsectionsfont{ \normalfont\scshape} % Make all sections the default font and small caps


\renewcommand{\thesubsection}{\alph{subsection}} %Make subsections start with letters

\usepackage{fancyhdr} % Custom headers and footers
\pagestyle{fancyplain} % Makes all pages in the document conform to the custom headers and footers
\fancyhead{} % No page header - if you want one, create it in the same way as the footers below
\fancyfoot[L]{} % Empty left footer
\fancyfoot[C]{} % Empty center footer
\fancyfoot[R]{\thepage} % Page numbering for right footer
\renewcommand{\headrulewidth}{0pt} % Remove header underlines
\renewcommand{\footrulewidth}{0pt} % Remove footer underlines
\setlength{\headheight}{13.6pt} % Customize the height of the header

\numberwithin{equation}{section} % Number equations within sections (i.e. 1.1, 1.2, 2.1, 2.2 instead of 1, 2, 3, 4)
\numberwithin{figure}{section} % Number figures within sections (i.e. 1.1, 1.2, 2.1, 2.2 instead of 1, 2, 3, 4)
\numberwithin{table}{section} % Number tables within sections (i.e. 1.1, 1.2, 2.1, 2.2 instead of 1, 2, 3, 4)

\setlength\parindent{0pt} % Removes all indentation from paragraphs - comment this line for an assignment with lots of text
\usepackage{listings}
\lstset{language=Python}


\usepackage{tikz}
\usetikzlibrary{decorations.pathmorphing}
\usetikzlibrary{arrows}


%----------------------------------------------------------------------------------------
%	TITLE SECTION
%----------------------------------------------------------------------------------------

\newcommand{\horrule}[1]{\rule{\linewidth}{#1}} % Create horizontal rule command with 1 argument of height

\title{	Homework 3}

\author{Benjamin Jakubowski} % Your name

\date{\normalsize\today} % Today's date or a custom date

\begin{document}

\maketitle % Print the title

%----------------------------------------------------------------------------------------
%	PROBLEM 1
%----------------------------------------------------------------------------------------

\section{Tail recursion}

\subsection{Tail recursive versions of \texttt{f} and \texttt{g}}

First, we present tail recursive versions of the methods \texttt{f} and \texttt{g}. We present the tail recursive versions in psuedocode (i.e. python).
 
\begin{lstlisting}
def f_tc(x, accum = 0):
    if x == 0:
        return accum
    else:
        return g_tc(x - 1, accum + 1)

def g_tc(x, accum = 0):
    if x == 0:
        return accum
    else:
        return f_tc(x - 1, accum + 2)
\end{lstlisting}

\subsection{Iterative version \texttt{fg\_iter}}

Now we present an iterative version of the method. Note in addition to the integer x, the method has an additional formal parameter \texttt{func}. This parameter indicates the function to apply (\texttt{f} or \texttt{g}) and as such the function must be passed either \texttt{'f'} or \texttt{'g'}.
\begin{lstlisting}
def fg_iter(x, func):
    accum = 0
    while x != 0:
        if func == 'f':
            accum = accum + 1
            func = 'g'
        elif func == 'g':
            accum = accum + 2
            func = 'f'
        else:
            raise ValueError
        x = x - 1
    return accum
\end{lstlisting}
%----------------------------------------------------------------------------------------
%	PROBLEM 2
%----------------------------------------------------------------------------------------

\section{Tail recursion}

Consider a free list with the following block size entries: 50, 20, 100, 50, 30, 60. When allocating a
memory block of size $n$ from a larger free block of size $k$, assume that a block of size $k - n$ will be placed
on the free list in the same cell as the original block. Further assume that if there are multiple blocks
of the same size on the free list satisfying an allocation request, the first satisfying block on the list will
be allocated.\\
Assume that allocation requests are made in the following order: 20, 20, 30, 50, 50, 45.\\

We show the free list under different memory allocation strategies:

\subsection{First fit}

\begin{center}
\begin{tabular} {| c | c |}
\hline
\textbf{Insert} & \textbf{Free list following insertion} \\
\hline
NA (Initial state) & [50, 20, 100, 50, 30, 60] \\
\hline
20& [30, 20, 100, 50, 30, 60] \\
\hline
20 & [10, 20, 100, 50, 30, 60] \\
\hline
30 & [10, 20, 70, 50, 30, 60] \\
\hline
50 & [10, 20, 20, 50, 30, 60] \\
\hline
50 & [10, 20, 20, 30, 60] \\
\hline
45 & [10, 20, 20, 30, 15] \\
\hline
\end{tabular}
\end{center}

\subsection{Best fit}

\begin{center}
\begin{tabular} {| c | c |}
\hline
\textbf{Insert} & \textbf{Free list following insertion} \\
\hline
NA (Initial state) & [50, 20, 100, 50, 30, 60] \\
\hline
20 & [50, 100, 50, 30, 60] \\
\hline
20 & [50, 100, 50, 10, 60] \\
\hline
30 & [20, 100, 50, 10, 60] \\
\hline
50 & [20, 100, 10, 60] \\
\hline
50 & [20, 100, 10, 10] \\
\hline
45 & [20, 55, 10, 10] \\
\hline
\end{tabular}
\end{center}

\subsection{Worst fit}

\begin{center}
\begin{tabular} {| c | c |}
\hline
\textbf{Insert} & \textbf{Free list following insertion} \\
\hline
NA (Initial state) & [50, 20, 100, 50, 30, 60] \\
\hline
20 & [50, 20, 80, 50, 30, 60] \\
\hline
20 & [50, 20, 60, 50, 30, 60] \\
\hline
30 & [50, 20, 30, 50, 30, 60] \\
\hline
50 & [50, 20, 30, 50, 30, 10] \\
\hline
50 & [20, 30, 50, 30, 10] \\
\hline
45 & [20, 30, 5, 30, 10] \\
\hline
\end{tabular}
\end{center}

\subsection{Additional 30 byte allocation}

Based on the final states of the free list after allocating 20, 20, 30, 50, 50, and 45 bytes of memory, all of the allocation strategies will support an additional 30 byte request.

%----------------------------------------------------------------------------------------
\end{document}